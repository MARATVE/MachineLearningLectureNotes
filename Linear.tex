\documentclass{article}


\usepackage{arxiv}

\usepackage[utf8]{inputenc} % allow utf-8 input
\usepackage[T1]{fontenc}    % use 8-bit T1 fonts
\usepackage{hyperref}       % hyperlinks
\usepackage{url}            % simple URL typesetting
\usepackage{booktabs}       % professional-quality tables
\usepackage{amsfonts}       % blackboard math symbols
\usepackage{nicefrac}       % compact symbols for 1/2, etc.
\usepackage{microtype}      % microtypography
\usepackage{lipsum}
\usepackage{graphicx}
\usepackage{outlines}
\usepackage{amsmath,amsthm,amssymb}
\usepackage{float}
\setlength{\parskip}{1em}
\setlength{\parindent}{0em} 
\newcommand{\N}{\mathbb{N}}
\newcommand{\Z}{\mathbb{Z}}
\newcommand{\Q}{\mathbb{Q}}
\newcommand{\R}{\mathbb{R}}
\newcommand{\C}{\mathbb{C}}
\newcommand{\F}{\mathbb{F}}
\newcommand{\V}{\mathcal{V}}
\newcommand{\U}{\mathcal{U}}
\newcommand{\W}{\mathcal{W}}
\newcommand{\K}{\mathbb{K}}
\newcommand{\PS}{\mathbb{S}}
\newcommand{\id}{\II}
\newcommand{\Pol}{\mathbb{P}}
\newcommand{\sls}{\mathcal{S}}
\newcommand{\vv}{{\bm{v}}}
\newcommand{\uu}{\bm{u}}
\newcommand{\cc}{\bm{c}}
\newcommand{\ww}{\bm{w}}
\newcommand{\ee}{\bm{e}}
\newcommand{\ff}{\bm{f}}
\newcommand{\hh}{\bm{h}}
\newcommand{\bfa}{\bm{a}}
\newcommand{\bfy}{\bm{y}}
\newcommand{\zz}{\bm{z}}
\newcommand{\pp}{\bm{p}}
\newcommand{\bfo}{\bm{0}} 
\newcommand\nulvec{\bm{0}}
\newcommand\nulmat{\bm{O}}
\newcommand\nulvecSet{\Set{\nulvec}}
\newcommand{\bfO}{\bm{O}}
\newcommand{\bfalp}{\bm{\alpha}}
\newcommand\la{\langle}
\newcommand\ra{\rangle}
\newcommand\E{\mathbb{E}}

\newcommand\Ein{E_{in}}
\newcommand\Eout{E_{out}}

\newenvironment{theorem}[2][Theorem]{\begin{trivlist}
\item[\hskip \labelsep {\bfseries #1}\hskip \labelsep {\bfseries #2.}]}{\end{trivlist}}
\newenvironment{lemma}[2][Lemma]{\begin{trivlist}
\item[\hskip \labelsep {\bfseries #1}\hskip \labelsep {\bfseries #2.}]}{\end{trivlist}}
\newenvironment{exercise}[2][Exercise]{\begin{trivlist}
\item[\hskip \labelsep {\bfseries #1}\hskip \labelsep {\bfseries #2.}]}{\end{trivlist}}
\newenvironment{problem}[2][Problem]{\begin{trivlist}
\item[\hskip \labelsep {\bfseries #1}\hskip \labelsep {\bfseries #2.}]}{\end{trivlist}}
\newenvironment{question}[2][Question]{\begin{trivlist}
\item[\hskip \labelsep {\bfseries #1}\hskip \labelsep {\bfseries #2.}]}{\end{trivlist}}
\newenvironment{corollary}[2][Corollary]{\begin{trivlist}
\item[\hskip \labelsep {\bfseries #1}\hskip \labelsep {\bfseries #2.}]}{\end{trivlist}}
\newcommand\arrows[3]{
        \begin{matrix}[c]
        \ifx\relax#1\relax\else \xrightarrow{#1}\fi\\
        \ifx\relax#2\relax\else \xrightarrow{#2}\fi\\
        \ifx\relax#3\relax\else \xrightarrow{#3}\fi
        \end{matrix}
}
\newcommand\inner[1]{\la#1\ra}
\newcommand\m[1]{\begin{pmatrix}#1\end{pmatrix}} 

\newcommand\vecnorm[1]{\lVert #1 \rVert}    
\newcommand{\lt}{\ensuremath <}
\newcommand{\gt}{\ensuremath >}




\title{Lecture Notes on linear models\\Machine Learning E20}


\author{
 Mathias Ravn Tversted \\
  Department of Computer Science\\
  Aarhus University\\
  Åbogade 34, 8200 Aarhus N \\
  \texttt{Tversted@post.au.dk} \\
  %% examples of more authors
  %% \AND
  %% Coauthor \\
  %% Affiliation \\
  %% Address \\
  %% \texttt{email} \\
  %% \And
  %% Coauthor \\
  %% Affiliation \\
  %% Address \\
  %% \texttt{email} \\
  %% \And
  %% Coauthor \\
  %% Affiliation \\
  %% Address \\
  %% \texttt{email} \\
}

\begin{document}
\maketitle
\tableofcontents
\newpage


\section{(Pocket) Perceptron Learning Algorithm}
  A way to nicely do \emph{linear classification} is to use the \emph{Perceptron Learning Algorithm}. 
  Let $X = \{1\} \times \R^d$ be the input space. Let $Y \in \{+1, -1\}$ be the output space. The coordinates of the $x$-vector corresponds to features of the input. We are looking to ``train'' weights and bias $w \in R^{d+1}$ such that $\sum_{i=0}^{d}{(\text{sign}(w_ix_i)\neq y_i)} = 0$ or, the data is perfectly linearly seperated. This is only possible if there exists some hypothesis that can actually do this.
  
  \subsection{The Perceptron Learning Algorithm}
    \begin{enumerate}
        \item Pick an $x$ s.t $y(t) \neq \text{sign}(w^T(t)x(t))$.
        \item \emph{Move the line in that direction} with update rule: $w(t+1) = w(t) + y(t)x(t)$
      \end{enumerate}
      This algorithm is \emph{guaranteed to terminate} given the data is linearly seperable.

      \subsection{Quick note: Pocket Perceptron Learning Algorithm}
        If the data is not linearly seperable, we can use the \emph{Pocket Perceptron Learning Algorithm}. We want to find the $w$ that minimises $E_{in}(w) = \frac{1}{n}\sum_{i=1}^{n}{1_{sign(x_i^Tw)\neq y_i}}$. 

        Run for set number of epochs, keep track of best. Return best solution after those epochs.

\section{Non-linear transforms}
      If the data is not linearly seperable, then we can transform the data to take advantage of potentially linearly seperable properties.

      A \emph{non-linear feature transform} is some $\Phi : R^d \rightarrow R^{d'}$ applied to all feature vectors to get some data-matrix $X' \in Mat_{n \times d'}$. 

      An obvious example of this is \emph{distance to origin} $\Phi(x_1, x_2) = (x_1^2 + x_2^2)$. 

      Learning theory tells us that simple hypotheses generalise better, and this is worth keeping in mind. 

\section{Regression}
    We will consider the problems of linear regression and logistic regression. Linear regression finds a best-fit linear model, and in logistic regression, we are working with probabilities instead of raw values.

      \subsection{Linear Regression}

      If instead of \emph{classification} we want to find \emph{values}, we can use \emph{regression}. Here we are also drawing some hyperplane, where we have a loss function (MSE) to be minimised $L(w) = \frac{1}{N}(Xw-y)^T(Xw-y)$. 

      % Tegn https://blackboard.au.dk/bbcswebdav/pid-2634429-dt-content-rid-9194690_1/courses/BB-Cou-UUVA-90726/Lec2Part2%281%29.pdf side 2/28 diagram

      Let $X \in \R^{N \times (d+1)}$ be the matrix of inputs, and $y \in \R^{n}$ be the target vector. 
      
      The in-sample error is MSE 
      \begin{align}
        & E_{in}(w) = \frac{1}{N}\sum_{n=1}^{N}{(w^Tx_n - y_2)^2} = \frac{1}{N}||Xw - y||^2
      \end{align}
      And we are looking to solve the following optimisation problem: 
      \begin{align}
       w_{lin} = \arg \min_{w \in \R^{d+1}}E_{in}(w)
      \end{align}

      We would like the gradient to be $\nabla l(w) = 0$. This is a column vector where $[\nabla l(w)]_i = \frac{\partial}{\partial w_i}l(w)$. 

     Note that 
     \begin{align}
     \frac{\partial}{\partial w_j}l(w) = \sum_{i=1}^{n} 2(x^T_iw - y_i)x_{i, j}
     \end{align}
     

      First we expand our loss function
      \begin{align}
        & (Xw - y)^T (Xw-y) = (w^TX^TXw + y^Ty - 2X^Tw^Ty)
      \end{align}

      Then we compute the derivative. We have the following identities that can help us:
      \begin{align}
        & \nabla_w(w^T \cdot A \cdot w)=(A+A^T) \cdot w && \text{Identity 1}\\
        & \nabla_w(w^T \cdot b) = b && \text{Identity 2} 
      \end{align}
    Yielding
      \begin{align}
        \nabla l(w) = \nabla (w^TX^TXw + y^Ty - 2w^TX^Ty) = 2(X^TXw-X^Ty)
      \end{align}
        We will then isolate the solution
        \begin{align}
            & 2(X^TXw - X^Ty) = 0 \leftrightarrow\\
            & (X^TXw - X^Ty) = 0 \leftrightarrow \\
            & X^TXw = X^Ty \leftrightarrow \\
            & w = (X^TX)^{-1}X^Ty
        \end{align}
        The R.H.S in (11) is the \emph{pseudoinverse}. Computing it gives us the least squares fit, or if $X^TX$ is invertible, a unique optimal solution.
      
      \subsubsection{The Linear Regression Algorithm}
      \begin{enumerate}
        \item Compute \emph{Pseudo-Inverse} $X^{\dagger}$.
        \item Return $w_{lin} = X^{\dagger}y$
      \end{enumerate}
      Here we have a nice \emph{analytic solution}. This gives us the ability to ``predict'' future values.
      
      The running time is 
      \begin{itemize}
          \item $X^TX$ is $O(d^2n)$
          \item Inverse $O(d^3)$
          \item $X^Ty$ is $O(dn)$
          \item $(X^TX)^{-1}(X^Ty)$ is $O(d^2)$
      \end{itemize}
      Giving us $O(d^2n + d^3)$

      \subsection{Logistic Regression}
      In \emph{logistic regression}, we are looking at probabilities and not values. Output values are thus bounded to the interval $[0, 1]$. We will use the \emph{logistic function} $\theta(s) = \frac{e^s}{1+e^s}$. As in our hypothesis $h$ is $h = \theta(w^Tx)$. This output is then interpreted as the \emph{probability for a binary event}, where we trying to learn a target function $f(x) = P[y = +1 | x]$, with us trying to find
      \begin{align}
        P(y | x) = \begin{cases}
          f(x) & \text{for } y = +1\\
          1 - f(x) & \text{for } y = -1
        \end{cases}
      \end{align}
      Because of potential noise.

      \subsubsection{Error measure:}
      We look at \emph{likelihood}. We want to find a hypothesis $h$ that maximises
      \begin{align}
        \prod_{n=1}^{N}{P(y_n | x_n)}
      \end{align}
      We prefer minimisation problems, and we want to be able to have an error measure, so we can transform this as follows:
      \begin{align}
        - \frac{1}{N} \ln{(\prod_{n=1}^{N}P(y_n | x_n))} = \frac{1}{N}\sum_{n=1}^{N}ln(\frac{1}{P(y_n | x_n)})
      \end{align}
      but the probabilities should be our $\theta$ function, thus giving us:
     \begin{align}
        E_{in}(w) = \frac{1}{N}\sum_{n=1}^{N}ln(\frac{1}{\theta(y_n w^Tx_n)}) = \frac{1}{N}\sum_{n=1}^{N}ln(1 + e^{-y_n w^T x_n})
     \end{align} 

     We can not expect to find a nice analytic solution to this problem, so we wil need an algorithm such as $\cdots$
\section{(Stochastic) Gradient Descent}
     Gradient descent allows us to minimise a twice-differentiable function. It allows us to think of $E_{in}(w)$ as a surface, where we can step in the direction of valleys. If the function is \emph{convex}, then it will find the global minimum, otherwise it will find a local minimum. 

    \textbf{Fixed learning rate gradient descent} 
    \begin{enumerate}
      \item Initialize the weights at step $t = 0$ to $w(0)$
      \item For $t = 0, 1, 2, \cdots $ \textbf{do}
      \item Compute gradient $g_t = \nabla E_{in}(w(t))$
      \item Set move direction $v_t = -g_t$
      \item Update weights $w(t+1) = w(t) + \eta v_t$
      \item Iterate until IT'S TIME TO STOP
    \end{enumerate}
    For logistic regression $\nabla E_{in}(w(t))$ equals $-\frac{1}{N}\sum_{n=1}^{N}\frac{y_nx_n}{1+e^{y_nw^T(t)x_n}}$

    \textbf{Initialization and termination}: Good in practice: Choose each weight independently from normal distribution (zero mean) and small variance.

    \subsection{Stochastic Gradient descent}
    We simply look at 1 point for each epoch instead of all the data. It is a much cheaper evaluation (factor N), but it is more jumpy. Eventually these erratic jumps cancel out.

\end{document}
