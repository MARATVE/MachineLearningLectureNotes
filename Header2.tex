\usepackage{outlines}
\usepackage{amsmath,amsthm,amssymb}
\usepackage{float}
\setlength{\parskip}{1em}
\setlength{\parindent}{0em} 
\newcommand{\N}{\mathbb{N}}
\newcommand{\Z}{\mathbb{Z}}
\newcommand{\Q}{\mathbb{Q}}
\newcommand{\R}{\mathbb{R}}
\newcommand{\C}{\mathbb{C}}
\newcommand{\F}{\mathbb{F}}
\newcommand{\V}{\mathcal{V}}
\newcommand{\U}{\mathcal{U}}
\newcommand{\W}{\mathcal{W}}
\newcommand{\K}{\mathbb{K}}
\newcommand{\PS}{\mathbb{S}}
\newcommand{\id}{\II}
\newcommand{\Pol}{\mathbb{P}}
\newcommand{\sls}{\mathcal{S}}
\newcommand{\vv}{{\bm{v}}}
\newcommand{\uu}{\bm{u}}
\newcommand{\cc}{\bm{c}}
\newcommand{\ww}{\bm{w}}
\newcommand{\ee}{\bm{e}}
\newcommand{\ff}{\bm{f}}
\newcommand{\hh}{\bm{h}}
\newcommand{\bfa}{\bm{a}}
\newcommand{\bfy}{\bm{y}}
\newcommand{\zz}{\bm{z}}
\newcommand{\pp}{\bm{p}}
\newcommand{\bfo}{\bm{0}} 
\newcommand\nulvec{\bm{0}}
\newcommand\nulmat{\bm{O}}
\newcommand\nulvecSet{\Set{\nulvec}}
\newcommand{\bfO}{\bm{O}}
\newcommand{\bfalp}{\bm{\alpha}}
\newcommand\la{\langle}
\newcommand\ra{\rangle}
\newcommand\E{\mathbb{E}}

\newcommand\Ein{E_{in}}
\newcommand\Eout{E_{out}}

\newenvironment{theorem}[2][Theorem]{\begin{trivlist}
\item[\hskip \labelsep {\bfseries #1}\hskip \labelsep {\bfseries #2.}]}{\end{trivlist}}
\newenvironment{lemma}[2][Lemma]{\begin{trivlist}
\item[\hskip \labelsep {\bfseries #1}\hskip \labelsep {\bfseries #2.}]}{\end{trivlist}}
\newenvironment{exercise}[2][Exercise]{\begin{trivlist}
\item[\hskip \labelsep {\bfseries #1}\hskip \labelsep {\bfseries #2.}]}{\end{trivlist}}
\newenvironment{problem}[2][Problem]{\begin{trivlist}
\item[\hskip \labelsep {\bfseries #1}\hskip \labelsep {\bfseries #2.}]}{\end{trivlist}}
\newenvironment{question}[2][Question]{\begin{trivlist}
\item[\hskip \labelsep {\bfseries #1}\hskip \labelsep {\bfseries #2.}]}{\end{trivlist}}
\newenvironment{corollary}[2][Corollary]{\begin{trivlist}
\item[\hskip \labelsep {\bfseries #1}\hskip \labelsep {\bfseries #2.}]}{\end{trivlist}}
\newcommand\arrows[3]{
        \begin{matrix}[c]
        \ifx\relax#1\relax\else \xrightarrow{#1}\fi\\
        \ifx\relax#2\relax\else \xrightarrow{#2}\fi\\
        \ifx\relax#3\relax\else \xrightarrow{#3}\fi
        \end{matrix}
}
\newcommand\inner[1]{\la#1\ra}
\newcommand\m[1]{\begin{pmatrix}#1\end{pmatrix}} 

\newcommand\vecnorm[1]{\lVert #1 \rVert}    
\newcommand{\lt}{\ensuremath <}
\newcommand{\gt}{\ensuremath >}

