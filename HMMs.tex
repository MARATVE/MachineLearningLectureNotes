\documentclass{article}


\usepackage{arxiv}

\usepackage[utf8]{inputenc} % allow utf-8 input
\usepackage[T1]{fontenc}    % use 8-bit T1 fonts
\usepackage{hyperref}       % hyperlinks
\usepackage{url}            % simple URL typesetting
\usepackage{booktabs}       % professional-quality tables
\usepackage{amsfonts}       % blackboard math symbols
\usepackage{nicefrac}       % compact symbols for 1/2, etc.
\usepackage{microtype}      % microtypography
\usepackage{lipsum}
\usepackage{graphicx}
\graphicspath{ {./images/} }
\usepackage{outlines}
\usepackage{amsmath,amsthm,amssymb}
\usepackage{float}
\setlength{\parskip}{1em}
\setlength{\parindent}{0em} 
\newcommand{\N}{\mathbb{N}}
\newcommand{\Z}{\mathbb{Z}}
\newcommand{\Q}{\mathbb{Q}}
\newcommand{\R}{\mathbb{R}}
\newcommand{\C}{\mathbb{C}}
\newcommand{\F}{\mathbb{F}}
\newcommand{\V}{\mathcal{V}}
\newcommand{\U}{\mathcal{U}}
\newcommand{\W}{\mathcal{W}}
\newcommand{\K}{\mathbb{K}}
\newcommand{\PS}{\mathbb{S}}
\newcommand{\id}{\II}
\newcommand{\Pol}{\mathbb{P}}
\newcommand{\sls}{\mathcal{S}}
\newcommand{\vv}{{\bm{v}}}
\newcommand{\uu}{\bm{u}}
\newcommand{\cc}{\bm{c}}
\newcommand{\ww}{\bm{w}}
\newcommand{\ee}{\bm{e}}
\newcommand{\ff}{\bm{f}}
\newcommand{\hh}{\bm{h}}
\newcommand{\bfa}{\bm{a}}
\newcommand{\bfy}{\bm{y}}
\newcommand{\zz}{\bm{z}}
\newcommand{\pp}{\bm{p}}
\newcommand{\bfo}{\bm{0}} 
\newcommand\nulvec{\bm{0}}
\newcommand\nulmat{\bm{O}}
\newcommand\nulvecSet{\Set{\nulvec}}
\newcommand{\bfO}{\bm{O}}
\newcommand{\bfalp}{\bm{\alpha}}
\newcommand\la{\langle}
\newcommand\ra{\rangle}
\newcommand\E{\mathbb{E}}

\newcommand\Ein{E_{in}}
\newcommand\Eout{E_{out}}

\newenvironment{theorem}[2][Theorem]{\begin{trivlist}
\item[\hskip \labelsep {\bfseries #1}\hskip \labelsep {\bfseries #2.}]}{\end{trivlist}}
\newenvironment{lemma}[2][Lemma]{\begin{trivlist}
\item[\hskip \labelsep {\bfseries #1}\hskip \labelsep {\bfseries #2.}]}{\end{trivlist}}
\newenvironment{exercise}[2][Exercise]{\begin{trivlist}
\item[\hskip \labelsep {\bfseries #1}\hskip \labelsep {\bfseries #2.}]}{\end{trivlist}}
\newenvironment{problem}[2][Problem]{\begin{trivlist}
\item[\hskip \labelsep {\bfseries #1}\hskip \labelsep {\bfseries #2.}]}{\end{trivlist}}
\newenvironment{question}[2][Question]{\begin{trivlist}
\item[\hskip \labelsep {\bfseries #1}\hskip \labelsep {\bfseries #2.}]}{\end{trivlist}}
\newenvironment{corollary}[2][Corollary]{\begin{trivlist}
\item[\hskip \labelsep {\bfseries #1}\hskip \labelsep {\bfseries #2.}]}{\end{trivlist}}
\newcommand\arrows[3]{
        \begin{matrix}[c]
        \ifx\relax#1\relax\else \xrightarrow{#1}\fi\\
        \ifx\relax#2\relax\else \xrightarrow{#2}\fi\\
        \ifx\relax#3\relax\else \xrightarrow{#3}\fi
        \end{matrix}
}
\newcommand\inner[1]{\la#1\ra}
\newcommand\m[1]{\begin{pmatrix}#1\end{pmatrix}} 

\newcommand\vecnorm[1]{\lVert #1 \rVert}    
\newcommand{\lt}{\ensuremath <}
\newcommand{\gt}{\ensuremath >}



\title{Lecture Notes on Hidden Markov Models\\Machine Learning E20}


\author{
 Mathias Ravn Tversted \\
  Department of Computer Science\\
  Aarhus University\\
  Åbogade 34, 8200 Aarhus N \\
  \texttt{Tversted@post.au.dk} \\
  %% examples of more authors
  %% \AND
  %% Coauthor \\
  %% Affiliation \\
  %% Address \\
  %% \texttt{email} \\
  %% \And
  %% Coauthor \\
  %% Affiliation \\
  %% Address \\
  %% \texttt{email} \\
  %% \And
  %% Coauthor \\
  %% Affiliation \\
  %% Address \\
  %% \texttt{email} \\
}

\begin{document}
\maketitle
\tableofcontents

\section{Definitions and terminology}

  \textbf{Hidden Markov Model}
  A \emph{Hidden Markov Model} consists of
  \begin{itemize}
    \item Transition Probability Matrix: $K \times K$ matrix $A$ that describe $K$ discrete states.
    \item Emission Probability Matrix: $K \times D$ matrix $\Phi$, where D is possible emissions/symbols. 
    \item Initialization Vector ($\pi$) with $K$ entries, one for each state.
  \end{itemize}   

  We will denote the set of latent states as $Z = \{z_1, \cdots, z_N\}$, the observables as $X = \{x_1, \cdots, x_N\}$ and model parameters as $\Theta = \{\pi, A, \phi\}$

  Where the following must apply
  \begin{itemize}
    \item The sum of any column k of $A$ always equal 1 because a column represents the outgoing transition probabilities of a state.
    \item The sum of any row k of $\Phi$ is equal 1, because a row represents a state with d emission possibilities.
    \item The sum of all indexes in $\pi$ is equal 1, because a chain must start in some initial state.
  \end{itemize}

  
  A Hidden Markov Model can be used to determine the likelihood of a sequence of observations. 
  It does this with a finite number of states, whose transition probabilities are kind of like a \emph{finite automata} ``encoding some process which is being modelled'', and where it can emit some output in each state.
  It is called a \emph{hidden} model if we do not know the probabilities, but are only able to observe the output of the process.

  Specifically, our joint distribution is 

  \begin{align}
    p(X, Z | \Theta) = p(z_1 | \pi) \left[
      \Pi_{n=2}^{N}{
        p(z_n | z_{n-1}, A)
      }
    \right]
    \Pi_{n=1}^{N}{
      p(x_n | z_n, \phi)
    }
  \end{align}
  If $A$ and $\Phi$ are the same for all $n$ then the HMM is \emph{homogeneous}. 
  
  Once all probabilities are in place, the HMM can also be used to generative that which it models. 
  I.E based on a model M - generate a most likely sequence of states from $M$, followed by a most likely sequence of emissions from these states.  


\section{Decoding algorithms}
  \subsection{Viterbi decoding}
    Given $X$, find $Z^*$ such that: 
    \begin{align}
      Z^* = \arg \max_Z p(X,Z|\Theta). 
    \end{align}
    Where $Z^*$ is the overall most likely explanation of $X$.

    We can express this as: 
    \begin{align}
      Z^*_N = \arg \max_{Z_N}\max_{Z_1,\dots,Z_{N-1}} p(x_1,\dots,x_N,Z_1,\dots,Z_N).
    \end{align}
    We express the last part of this expression as
    \begin{align}
      \omega(Z_N) = \max_{Z_1,\dots,Z_{N-1}} p(x_1,\dots,x_N,Z_1,\dots,Z_N) 
    \end{align}
    which is what we wish to maximize. So $\omega$ is the probability of the most likely sequence of states $z_1,\dots,x_n$ ending in $z_n$ given the observations $x_1,\dots,x_n$.


    We can calculate $\omega(Z_n)$ through recursion

      \begin{itemize}
        \item \textbf{General case}: $\omega(Z_n) = p(x_n|Z_n)\max_{Z_{n-1}}p(Z_n|Z_{n-1})\omega(Z_{n-1})$
        \item \textbf{Base case:} $\omega(Z_1) = p(x_1,z_1) = p(z_1)p(x_1|z_1)$
      \end{itemize}

    We use this to calculate a matrix $\Omega$ which can be used for backtracking. 

    Computing $\Omega$ takes $O(K^2N)$ with $O(KN)$ space with memoization.

    \textbf{Backtracking Algorithm}
    \begin{outline}
      \1 $Z[N] = \arg \max_k \omega[k][N]$
      \1 for $n = N-1$ to 1:
        \2 $Z[n] = \arg \max_k (p(x[n+1] | Z[n+1]) \cdot \omega[k][n] \cdot p(Z[n+1] | k))$
    \end{outline}
    
    This algorithm runs in $O(KN)$ time.

    When we backtrack we start in the last column of the matrix and select the state that has the maximum value.
    Then we find the state that was most likely in the $Z_{n-1}$ step. i.e. what gives get largest $\omega$ for the state in the column that we picked $Z_{n}$.

\section{Training by counting}
  
  We can count how many times the transition from $j$ to $k$ is taken, over how many times a transition from state $j$ is taken.
  \begin{align}
    A_{jk} = \frac{
      \sum_{n=2}^{N}z_{n-1,j} z_{nk}
    }{
      \sum_{l=1}^{K}\sum_{n=2}^{N}{z_{n-1, j}z_{nl}}
    }
  \end{align}
  With \emph{pseudo-counting} we assume that every transition has been seen at least once.

  \begin{align}
    \pi_k = \frac{z_1k}{\sum_{j=1}^{K}{z_{1j}}}
  \end{align}
  We can take how many times a symbol $i$ is emitted from state $k$ over how many times a symbol os emitted from state $k$. 
  \begin{align}
    \phi_{ik} = \frac{
      \sum_{n=1}^{N}z_{nk}x_{ni}
    }{
      \sum_{n=1}^{N}{z_{nk}}
    }
  \end{align}

  This yields a \emph{maximum likelihood estimate} (MLE) $\Theta^*$ of $P(X, Z | \Theta)$. This is known as \emph{training by counting} and is a nice analytic solution.


  \section{Viterbi Training}

  If the latent states $Z$ are unknown, but we only have a sequence $X$ of observations, then we are looking to solve
  \begin{align}
    \max p(X |\Theta) = \sum_{Z}{p(X, Z | \Theta)} w.r.t \Theta
  \end{align}
  This is a hard problem, and to solve it we will use \emph{Viterbi Training.}

  \textbf{Viterbi Training}

  \begin{enumerate}
    \item Decide on initial parameter $\Theta^0$
    \item Compute most likely sequence of states $Z^*$ under $X$ using the Viterby Algorithm, yielding $\Theta^i$
    \item Update to $\Theta^{i+1}$ by pseudocounting according to $(X, Z^*)$
    \item Repeat 3-4 until $P(X, Z^* | \Theta^i)$ makes you happy, or the sequence of states does not change
  \end{enumerate}
  This yields $\text{VIT}_X(\Theta) = \max_{Z}p(X, Z | \Theta)$. This is NOT an MLE, but it works.

  $p(X | \Theta_{\text{Vit}}^i)$ is not guaranteed, but usually is, close to $p(X | \Theta^*_X)$




  \subsection{Posterior decoding} 
  $z^*_n$ is the most likely state to be in the $n$'th step. 
  \begin{align}
    z^*_n = \arg \max_{z_n} p(z_n | x_1, \dots, x_N)
  \end{align}

  We have that
   \begin{align}
    p(z_n | x_1, \dots, x_N)  = \frac{p(x_1, \dots, x_n, z_n) p(x_{n+1}, \dots, x_N | z_n)}{p(x_1, \dots, x_N)}  = \frac{\alpha(z_n) \beta(z_n)}{p(X)}
  \end{align}
  Where $\alpha$ is the joint probability of observering up to $x_n$ and being in state $z_n$, and $\beta$ is the conditional probability of future observations from $x_{n+1}$ up to $x_N$ if we are in state $z_n$. 

  In total, this gives us that
  \begin{align}
    z^*_n = \arg \max_{z_n} p (z_n | x_1, \dots, x_N) = \arg \max_{z_n} \alpha{z_n}\beta{z_n}/p(X)
  \end{align}
  
  Which is found by calculating the value for all of the states $Z$. This is calculated in $O(K^2N)$ time bounded by $\alpha$ and $\beta$.

  We also have that $p(X) = \sum_{z_n \in Z}{\alpha(z_n)}$

\subsection{Calculating $\alpha$ and $\beta$}
  The $\alpha$ values are also called \emph{Forward values}.
  They can be calculated recursively.
  \begin{itemize}
    \item \textbf{General case:} $\alpha(z_n) = p(x_n | z_n) \sum_{z_{n-1}}p(z_n | z_{n-1})\alpha(z_{n-1})$
    \item \textbf{Base case:} $\alpha(z_1) = p(x_1, z_1) = p(z_1)p(x_1|z_1)$
  \end{itemize}
  
  The $\beta$ values, also called \emph{Backward} values.
  \begin{itemize}
    \item \textbf{General case:} $\beta(z_n) = \sum_{z_{n-1}}\beta(z_{n-1})p(x_{n+1}|z_{n+1})p(z_{n+1}|z_n)$
    \item \textbf{Base case:} $\beta(z_N) = 1$
  \end{itemize}
  
  
  






  \section{Expectation Maximisation}
        We are given a sequence of observations $X$. Pick an initial set of parameters $\theta^0_{EM}$ and consider the expectation of the log-likelihood $\log p(X, Z | \Theta)$ over $Z$ (given $X$ and $\theta^0_{EM}$) as function of $\Theta$. 
        
        Because direct maximisation of log-likelihood is hard, we can, using an iterative method, consisting of an \emph{expectation step} and a \emph{maximisation step}, compute better and better estimates of $\Theta$. 
        
      Define the $Q$-function as 
      \begin{align}
        Q(\Theta, \theta^{old}) = \sum_{Z}p(Z | X, \Theta^{old}) logp(X, Z | \Theta)
      \end{align}
        Recall definition of entropy. 
        
        In other words, we have a valid set of parameters $\Theta^{old}$ and want to estimate a set $\Theta$ which yields a better likelihood.
      
        Specifically, if
        \begin{align}
            EM_{X, \Theta^0_{EM}}(\Theta) = E_{Z | X, \Theta^0_{EM}} = \sum_{Z}(p(Z | X, \Theta^0_{EM}) \log p(X, Z | \Theta)
        \end{align}
        
        Specifically, we are iterating through the following
        \begin{align}
            \Theta^{t+1} = \arg \max_{\Theta} Q(\Theta, \Theta^{t})
        \end{align}
        It turns out that (proof omitted) these better and better values can be computed with the forward and backwards algorithms. 
        
        This gives rise to our algorithm:
        
      \subsubsection{EM algorithm}
      
        \begin{enumerate}
            \item \textbf{Init:} Pick suitable non-zero parameters (zero remains zero)
            \item \textbf{E-Step} Run forward and backward algorithms for current choice of parameters to get parameters of $Q$-function, or, enable us to construct $\Theta$s.
            \item \textbf{M-Step}: Compute $\Theta^{t+1}$
            \item Compute (log) likelihood $p( X | \Theta)$ and use this to determine if you are happy.
        \end{enumerate}
        
        Running time is $O(K^2N + KK + K^2NK + KDN)$ per iteration, or $O(K^2N + KDN)$ with memorization, and it (without guarantees) converges towards a (local) maximum of $p( X | \Theta)$.
        
        
        
\section{Extensions}
    Improve performance of a HMM using specialized architectures.
    Autoregressive HMMs for genes - consider emissions from $x_{n-1}$, $x_{n-2}$, when calculating emission from $x_n$. Each emission becomes a codon.    
    
    
    \subsection{Logspace and scaling}
  To avoid issues with really small values, we can compute $\log(\omega(z_n))$. Because log is monotonically increasing, it preserves inequalities. It also turns products into sums. If a factor is $0$, the product becomes $0$. We turn $log(0)$ into a representation of ``minus infinity''. This works because we are using products.
  This new transformed $\omega$ gets denoted as $\hat{\omega}$ and can also be calculated using recursion as so:
  \begin{itemize}
    \item $\hat{\omega}(z_n) = \log p(x_n | z_n) + \max_{j=1}^{n-1}(\hat{\omega}(z_j)+\log p(z_n | z_j))$
  \end{itemize}
  The Basis is:
  \begin{itemize}
    \item $\hat{\omega}(z_1) = \log(p(z_1)p(x_1|z_1)) = \log p(z_1) + \log p(x_1 | z_1)$
  \end{itemize}


  We can't use log transform for $\alpha$ and $\beta$ since $\log(\Sigma f) \neq \Sigma(\log f)$.

  Instead, we calculate a scaled (normalized) version of $\alpha$ called $\hat{\alpha}$ using recursion like so. We calculate the following in each step.
  \begin{itemize}
    \item $\delta(z_n) = p(x_n|z_n)\sum_{j=1}^{n-1}\hat{\alpha}(z_j)p(z_n|z_j)$
    \item $c_n = \sum_{k=1}^K\delta(z_{nk})$
    \item $\hat{\alpha}(z_{nk}) = \delta(z_{nk})/c_n$
  \end{itemize}
  The Basis is:
  \begin{itemize}
    \item $\hat{\alpha}(z_1) = \alpha(z_n)$
    \item $c_1 = \sum_{k=1}^K\pi_k p(x_1|\phi_k)$
  \end{itemize}

  The $\alpha$s  has the following relation to $\hat{\alpha}$: $\alpha(z_n) = \left(\prod_{m=1}^n c_m \right) \hat{\alpha}(z_n)$

  $\beta$ can not be scaled using log transform neither. So here we also make a scaling.
  $\hat{\beta}$ is calculated using recursion as so ($c_{n+1}$ is the c from the forward recursion):\\
  \begin{itemize}
    \item $\epsilon(z_n) = \sum_{j=1}^{n-1}\hat{\beta}(z_j)p(x_{n+1} | z_j)p(z_j | z_n)$
    \item $\hat{\beta}(z_nk) = \epsilon(z_nk)/c_{n+1}$
  \end{itemize}
  The Basis is:
  \begin{itemize}
    \item $\hat{\beta}(z_N) = 1$
  \end{itemize}


\end{document}
